\documentclass[a4paper,12pt]{article}
%\documentclass[a4paper,12pt]{scrartcl}

\usepackage{xltxtra}

\input{../preamble.tex}

\usepackage[spanish]{babel}

% \setromanfont[Mapping=tex-text]{Linux Libertine O}
% \setsansfont[Mapping=tex-text]{DejaVu Sans}
% \setmonofont[Mapping=tex-text]{DejaVu Sans Mono}

\title{Tarea \# 01: Probabilidad y estadística en las ciencias sociales}
\author{Isaac Ayala Lozano}
\date{2020-01-14}

\begin{document}
\maketitle

El impacto de la estadística en las ciencias sociales puede ser rastreado hasta Francis Galton \cite{galton1883inquiries} con sus contribuciones hacia el campo de la eugenesia, del cual él es uno de los pioneros.
En sus publicaciones sobre estudios de la población, introduce metodologías para el análisis de rasgos hereditarios en poblaciones humanas.
Es esta práctica la que da paso al uso de la estadística en las ciencias sociales.

No obstante, el estudio de la población data de tiempos mucho más remotos \cite{courgeau2012probability}. 
La necesidad de conocer información cuantitativa de una población ha sido siempre una característica de todo gobierno.
La toma de decisiones informada puede suceder solamente si se cuenta con mediciones actualizadas, permitiendo así cambios benéficos para el pueblo.

Observando la evolución de las ciencias sociales junto con la estadística, se observa que es la estadística quien provee a las ciencias sociales de las herramientas adecuadas para el estudio de poblaciones.
Al considerar a las acciones del ser humano como eventos difíciles de estudiar con métodos determinísticos \cite{courgeau2012probability} el uso de modelos probabilísticos es el estándar.

Los efectos de la estadística en las ciencias sociales puede ser percibido fácilmente. 
Ejemplos claros de esto se encuentran en la política y la economía, donde la el uso de encuestas y predicciones tienen un impacto impresionante en el comportamiento de grupos humanos.
Encuestas de popularidad, predicciones sobre el valor de empresas y monedas, así como estudios del impacto de leyes en el sector privado tienen todos efectos considerables en los seres humanos.


\printbibliography

\end{document}
