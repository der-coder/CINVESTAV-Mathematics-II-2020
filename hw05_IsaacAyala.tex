\documentclass[a4paper,12pt]{article}
%\documentclass[a4paper,12pt]{scrartcl}

\usepackage{xltxtra}

\input{../preamble.tex}

% \usepackage[spanish]{babel}
% \decimalpoint

% \setromanfont[Mapping=tex-text]{Linux Libertine O}
% \setsansfont[Mapping=tex-text]{DejaVu Sans}
% \setmonofont[Mapping=tex-text]{DejaVu Sans Mono}

\title{Homework \#05: Continuous Random Variable}
\author{Isaac Ayala Lozano}
\date{2020-02-03}

\begin{document}
\maketitle

\section{Problem}
\begin{enumerate}
 \item Plot the probability distribution functions for the Exponential and Gamma Distributions.
 Consider $\lambda \in \{0.5, 1, 2\}$ for the Exponential Distribution and $\alpha \in \{2, 3, 7, 7\}, \lambda \in \{0.5, 2, 2, 1\}$ for the Gamma Distribution.
 \item Provide proof that the Gamma distribution is indeed a probability density function.
 \item Relate the Uniform, Exponenial and Gamma Distributions to physical phenomena or engineering problems.
\end{enumerate}

\pagebreak

\section{Solution}

Plots are presented for the Probability Density Function of the Exponential (figure \ref{fig: exp}) and Gamma (figure \ref{fig: gamma}) Distributions.


\begin{figure}[htb!]
\centering
\import{./img/}{hw05_exp.tex}
\caption{Exponential Distribution.}
\label{fig: exp}
\end{figure}

\begin{figure}[htb!]
\centering
\import{./img/}{hw05_gamma.tex}
\caption{Gamma Distribution.}
\label{fig: gamma}
\end{figure}

\pagebreak

We provide proof that the Gamma Distribution probability function is indeed a probability density function.
From \cite{stewart2009probability}, the Gamma Distribution is defined as

\begin{equation}
 f(x) \equiv f(x;\beta, \alpha) = \dfrac{1}{\beta^\alpha \Gamma(\alpha)} x^{\alpha-1} \exp(-x/\beta), \quad x>0
\end{equation}

By establishing $\alpha = r$ and $\beta = 1/ \mu$, the function can be rewritten as

\begin{equation}
 f(x) = \dfrac{\mu (\mu x)^{r-1} \exp(-\mu x)}{\Gamma(r)}
\end{equation}

this form is also known as the \emph{Erlang-r} distribution.
Integrating this new expression we have the following

\begin{align}
 \int_0^\infty f(x)dx &= \int_0^\infty \dfrac{\mu (\mu x)^{r-1} \exp(-\mu x)}{\Gamma(r)}dx\\
 u &= \mu x \\
 du &= \mu dx\\
 &= \dfrac{1}{\Gamma(r)} \int_0^\infty (u)^{r-1} \exp(-u) du
\end{align}

Recall the definition of the Gamma function

\begin{equation}
 \Gamma(\alpha) = \int_0^\infty y^{\alpha-1}exp(-y) dy
\end{equation}

Thus the equation is simplified as

\begin{align}
 \int_0^\infty f(x)dx &= \dfrac{\Gamma(r)}{\Gamma(r)}\\
 &= 1
\end{align}


We present examples of the three distributions studied.

\begin{itemize}
 \item Uniform Distribution. Events in which all results have the same probability of occurring such as the odds of any number of a six sided die being rolled or the result of a coin toss.
 \item Exponential Distribution. Events related to time and events, such as number of number of calls received during an hour, rainfall in a year, distribution of gas molecules at fixed temperature and pressure.
 \item Gamma Distribution. Also used for events related to time and rate, such as the amount of customers arriving to a restaurant in a specific time interval. Consider a restaurant where customers arrive at x customer per hour, what is the probability of k customers arriving in a given time interval?
\end{itemize}


\printbibliography

\pagebreak
\appendix
\section{Octave Code}
\lstinputlisting[language=Matlab]{hw05_plots.m}


\end{document}
