\documentclass[a4paper,12pt]{article}
%\documentclass[a4paper,12pt]{scrartcl}

\usepackage{xltxtra}

\input{../preamble.tex}

% \usepackage[spanish]{babel}

% \setromanfont[Mapping=tex-text]{Linux Libertine O}
% \setsansfont[Mapping=tex-text]{DejaVu Sans}
% \setmonofont[Mapping=tex-text]{DejaVu Sans Mono}

\title{Homework \#04: Random Variable and Discrete Distribution Functions}
\author{Isaac Ayala Lozano}
\date{2020-01-27}

\begin{document}
\maketitle

\section{Problem}
\begin{enumerate}
 \item Plot the behaviour of the Binomial, Geometric and Poisson distributions for the following values
    \begin{itemize}
     \item Binomial distribution: $n \in [10, 20, 30]$ and $p \in [0.1, 0.3, 0.5]$
     \item Geometric distribution: $p \in [0.1, 0.3, 0.5]$
     \item Poisson distribution: $\lambda \in [1, 2, 3]$
    \end{itemize}
\item Show that the Poisson distribution is a good approximation to the Binomial distribution for large values of n and small values of p.
\end{enumerate}

\section{Solution}

\subsection{Plots}
Figure \ref{fig: binomial plots} shows the plots for the Binomial distribution.\\

\begin{figure}[htb!]
\centering
\import{./img/}{hw04_binomial.tex}
\caption{Binomial distribution.}
\label{fig: binomial plots}
\end{figure}

\pagebreak

Figure \ref{fig: geometric plots} shows the plots for the Geometric distribution.\\

\begin{figure}
\centering
\import{./img/}{hw04_geometric.tex}
\caption{Geometric distribution}
\label{fig: geometric plots}
\end{figure}

\pagebreak

Figure \ref{fig: poisson plots} shows the plots for the Poisson distribution.\\

\begin{figure}[htb!]
\centering
\import{./img/}{hw04_poisson.tex}
\caption{Poisson distribution}
\label{fig: poisson plots}
\end{figure}

\pagebreak

\subsection{Proof}

From \cite{ross2006introduction} we obtain the equations for the Binomial and Poisson distributions, equations \eqref{eq: binomial} and \eqref{eq: poisson} respectively.

\begin{equation}
P (X) = \binom n i p^i (1 -p)^{n-i}
\label{eq: binomial}
\end{equation}

\begin{equation}
 P (X) = \exp(-\lambda) \dfrac{\lambda ^i}{i !}
 \label{eq: poisson}
\end{equation}

Let $\lambda = n p$ such that \eqref{eq: binomial} can written as follows

\begin{equation*}
 \binom n i p^i (1 -p)^{n-i} = \dfrac{n!}{(n-i)!  i!} \left (\frac{\lambda}{n} \right )^i \left(1 -\dfrac{\lambda}{n}\right)^{n-i}
\end{equation*}

Rewrite the last term of the equation.

\begin{equation}
 \dfrac{n!}{(n-i)!  i!} \left (\frac{\lambda}{n} \right )^i \dfrac{(1 - \lambda / n  )^n}{(1 - \lambda / n  )^i}
 \label{eq: binomial to poisson fraction}
\end{equation}

Recall that for large values of n the term $(1 - \lambda / n  )^n $ is approximately equal to $ exp(-\lambda)$.
Thus the equation above can be expressed as

 \begin{equation}
  \dfrac{n!}{(n-i)!  i!} \left (\frac{\lambda}{n} \right )^i \dfrac{exp(-\lambda)}{(1 - \lambda / n  )^i}
   \label{eq: binomial to poisson exp}
 \end{equation}

 The term $ \dfrac{n!}{(n-i)!  i!}$ can be expanded as follows

 \begin{equation*}
   \dfrac{n!}{(n-i)!  i!} =  \dfrac{n * (n-1)!}{(n-i)!  i!}
   =  \dfrac{n*(n-1)*(n-2)!}{(n-i)!  i!}
 \end{equation*}

 This expansion permits us to rewrite the term as

 \begin{equation*}
  \dfrac{n!}{(n-i)!  i!} =  \dfrac{n*(n-1)* (n-2) \dots (n-i+1)* (n-i)!}{(n-i)!  i!}
 \end{equation*}

 Notice how this new expression permits us to eliminate  $(n-i)!$ from the equation.

 \begin{equation*}
  \dfrac{n!}{(n-i)!  i!} = \dfrac{n*(n-1)* (n-2) \dots (n-i+1)}{i!}
 \end{equation*}

 Equation \eqref{eq: binomial to poisson exp} can now be rewritten as

 \begin{equation}
   \dfrac{n(n-1) (n-2) \dots (n-i+1)}{i!} \dfrac{\lambda ^i}{n ^i} \dfrac{exp(-\lambda)}{(1 - \lambda / n  )^i}
 \end{equation}


 For large values of n, the term $(1 - \lambda / n  )^i$ is approximately equal to one.
 This simplifies the equation once again.

 \begin{equation}
   \dfrac{n(n-1) (n-2) \dots (n-i+1)}{i!} \dfrac{\lambda ^i}{n ^i} exp(-\lambda)
 \end{equation}

 Exchanging the denominators in the fractions we obtain

 \begin{equation}
   \dfrac{n(n-1) (n-2) \dots (n-i+1)}{n^i} \dfrac{\lambda ^i}{i!} exp(-\lambda)
   \label{eq: binomial to poisson denominators}
 \end{equation}

 Expanding the polynomial equation from the terms $n(n-1)\dots (n-i+1)$

 \begin{equation*}
  \begin{cases}
   n (n-1) &= n^2 -n \\
   n (n-1) (n-2) &= n^3 -3 n^2 + 2n \\
   \vdots & \\
   n (n-1) (n-2) \dots (n-i+1) &= n^i - i n^{i-1} + (i-1)n^{i-2} \dots +  n
  \end{cases}
 \end{equation*}

 Rewriting the first term of the equation and evaluating the limit as n goes to infinity we have

 \begin{equation}
\lim_{n\to\infty}  \dfrac{n^i - \beta_1 n^{i-1} + \beta_2 n^{i-2} \dots +  \beta_n n}{n^i} = 1
 \end{equation}

 Equation \eqref{eq: binomial to poisson denominators} is now the equation for the Poisson distribution due to the result of the limit shown above.
 We conclude that for large values of n and small values of p, the Poisson distribution can approximate the result of the Binomial distribution.

 \begin{equation}
  P(X) = \binom n i (p)^i (1-p)^{n-i} \approx \dfrac{\lambda ^i}{i!} exp(-\lambda)
 \end{equation}


\printbibliography

\pagebreak
\appendix
\section{Código Octave}
\lstinputlisting[language=Matlab]{hw04_plots.m}

\end{document}
