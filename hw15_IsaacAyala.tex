\documentclass[a4paper,12pt]{article}
%\documentclass[a4paper,12pt]{scrartcl}

\usepackage{xltxtra}

\input{../preamble.tex}

% \usepackage[spanish]{babel}

% \setromanfont[Mapping=tex-text]{Linux Libertine O}
% \setsansfont[Mapping=tex-text]{DejaVu Sans}
% \setmonofont[Mapping=tex-text]{DejaVu Sans Mono}

\title{Homework \#15}
\author{Isaac Ayala Lozano}
\date{\today}

\begin{document}
\maketitle

\begin{enumerate}
 \item Explain why $G_{xx}(f)$ is a constant for Example 6.3 in \cite{bendat2011random}.

 The function $G_{xx}(f)$ is a constant for the example because $G_{xx}(f)$ is the autospectral density function of the input.
 This function is defined as

 \begin{equation*}
  G_{xx} = 2 S_{xx}(f)
 \end{equation*}

 Recall that

\begin{equation*}
 S_{xx}(f) = 2 \int_0^\infty R_{xx}(\tau) \cos(2 \pi f \tau ) d\tau
\end{equation*}

 As such, $G_{xx}$ becomes

 \begin{equation*}
  G_{xx} = 4 \int_0^\infty R_{xx}(\tau) \cos(2 \pi f \tau ) d\tau
 \end{equation*}


 And $R_{xx}$ is defined as

\begin{equation*}
 R_{xx}(\tau) = E[x_k(t) x_k(t+\tau)]
\end{equation*}

The expected value of the input, white noise, is a constant $\mu$ that depends on the samples. Thus, $G_{xx}$ is a constant because the Fourier transformation of the original input results in a constant as well.

\newpage

\item Explain how the output spectral function $G_{yy}$ for Example 6.3 results in the output autocorrelation function $R_{yy}$ shown below.

\begin{align*}
 G_{yy}(f) &= \abs{H(f)}_{f-d}^2 G = \frac{G}{\abs{1-(f/f_n)^2}^2 + (2 \zeta f/f_n)^2 } \quad 0 \leq f < \infty \\
 R_{yy}(\tau) &= \frac{G \pi f_n \exp(-2\pi f_n \zeta \abs{\tau})}{4 \zeta} F(\tau, \zeta) \\
 F(\tau, \zeta) &=
  \cos(2 \pi f_n \sqrt{1-\zeta^2} \abs{\tau})
 +
 \frac{\zeta}{\sqrt{1-\zeta^2}}
 \sin(2 \pi f_n \sqrt{1-\zeta^2} \abs{\tau})
\end{align*}

Recall that the the autocorrelation function can be expressed as

\begin{equation*}
 R_{yy}(\tau) = \int_0^\infty G_{yy} (f) \cos(2 \pi f \tau) df
\end{equation*}

Substituting known values in the expression

\begin{align*}
 R_{yy}(\tau) &= \int_0^\infty \abs{H(f)}_{f-d}^2 G \cos(2 \pi f \tau) df \\
 &= G \int_0^\infty \abs{H(f)}_{f-d}^2 \cos(2 \pi f \tau) df \\
 &= G \int_0^\infty
 \frac{\cos(2 \pi f \tau)  }{\abs{1-(f/f_n)^2}^2 + (2 \zeta f/f_n)^2 }
 df
\end{align*}

Solving the integral, we obtain the following

\begin{align*}
 R_{yy}(\tau) &= \frac{G \pi f_n \exp(-2\pi f_n \zeta \abs{\tau})}{4 \zeta}
 \cos(2 \pi f_n \sqrt{1-\zeta^2} \abs{\tau}) \\
 & \qquad + \frac{G \pi f_n \exp(-2\pi f_n \zeta \abs{\tau})}{4 \zeta}
 \frac{\zeta}{\sqrt{1-\zeta^2}}
 \sin(2 \pi f_n \sqrt{1-\zeta^2} \abs{\tau})
\end{align*}

which can then be factorized to

\begin{align*}
  R_{yy}(\tau) &= \frac{G \pi f_n \exp(-2\pi f_n \zeta \abs{\tau})}{4 \zeta} F(\tau, \zeta) \\
 F(\tau, \zeta) &=
  \cos(2 \pi f_n \sqrt{1-\zeta^2} \abs{\tau})
 +
 \frac{\zeta}{\sqrt{1-\zeta^2}}
 \sin(2 \pi f_n \sqrt{1-\zeta^2} \abs{\tau})
\end{align*}

\newpage

\item Present plots for both $G_{yy}$ and $R_{yy}$

\begin{figure}[htb!]
\centering
\import{./img/}{hw15_comparison.tex}
\caption{Comparison.}
\label{fig: comparison}
\end{figure}

\begin{figure}[htb!]
\centering
\import{./img/}{hw15_Gyy.tex}
\caption{Output Spectral function.}
\label{fig: gyy}
\end{figure}


\begin{figure}[htb!]
\centering
\import{./img/}{hw15_Ryy.tex}
\caption{Output Autocorrelation function.}
\label{fig: ryy}
\end{figure}

\end{enumerate}


\newpage
\pagebreak

\printbibliography

\newpage
\pagebreak
\appendix
\section{Octave Code}
\lstinputlisting[language=Matlab]{hw15_plots.m}

\end{document}
