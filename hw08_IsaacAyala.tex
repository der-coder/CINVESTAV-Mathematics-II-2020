\documentclass[a4paper,12pt]{article}
%\documentclass[a4paper,12pt]{scrartcl}

\usepackage{xltxtra}

\input{../preamble.tex}

% \usepackage[spanish]{babel}

% \setromanfont[Mapping=tex-text]{Linux Libertine O}
% \setsansfont[Mapping=tex-text]{DejaVu Sans}
% \setmonofont[Mapping=tex-text]{DejaVu Sans Mono}

\title{Homework \#08: Collective wisdom}
\author{Isaac Ayala Lozano}
\date{2020-03-02}

\begin{document}
\maketitle

An example of how the estimate for the total number of jelly beans in a jar approximates the true value as the number of samples increases is presented.
The study considers a jar with 1500 jelly beans stored inside of it, with a variance of $\sigma^2 = 1000$.
Multiple samples are considered.
Figure \ref{fig: hist} shows how the group approaches the true value as the sample size increases.
The histograms presented show the estimated values after being normalised for ease of comparison.

\begin{figure}[htb!]
\centering
\import{./img/}{hw08_hist.tex}
\caption{Histograms.}
\label{fig: hist}
\end{figure}

% \printbibliography

\newpage
\pagebreak
\appendix
\section{Octave Code}
\lstinputlisting[language=Matlab]{hw08_plots.m}

\end{document}
