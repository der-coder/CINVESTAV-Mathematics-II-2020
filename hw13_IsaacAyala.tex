\documentclass[a4paper,12pt]{article}
%\documentclass[a4paper,12pt]{scrartcl}

\usepackage{xltxtra}

\input{../preamble.tex}
\usepackage[most]{tcolorbox}

% \usepackage[spanish]{babel}

% \setromanfont[Mapping=tex-text]{Linux Libertine O}
% \setsansfont[Mapping=tex-text]{DejaVu Sans}
% \setmonofont[Mapping=tex-text]{DejaVu Sans Mono}

\title{Homework \#13}
\author{Isaac Ayala Lozano}
\date{\today}

\begin{document}
\maketitle

\textbf{Problem.} Explain why a system can be completely characterized with its impulse response.

\begin{tcolorbox}[colback=black!5!white,colframe=black!75!black]
  A LTI system is fully characterized by knowing its impulse response $h(t)$ because any input to the system can be represented as a sum of impulses. Likewise, the output of the system can also be represented as the sum of all impulse responses corresponding to the inputs.

  \begin{equation}
   x (t) = \sum_{i = 0}^M \alpha_i \delta(t-k_i) \rightarrow y(t) = \sum_{i = 0}^M \alpha_i h(t-k_i)
  \end{equation}

\end{tcolorbox}

We begin by assuming that the system $\mathfrak{F}$ is a Linear Time Invariant (LTI) system.
As such, the superposition principle applies.
This implies that the system has the following properties:
\begin{align*}
 \mathfrak{F} (x_1 + x_2) = \mathfrak{F}(x_1) + \mathfrak{F}(x_2)  & \quad \text{Additivity}\\
 \mathfrak{F}(\alpha \, x) = \alpha \mathfrak{F}(x)& \quad \text{Homogeneity}
\end{align*}


For a given system $\mathfrak{F}$, the output of the system is defined as $y = \mathfrak{F}\{x\}$.
For an input $x_n$, there will be a corresponding output $y_n$.

Consider now the unit impulse function $\delta$.
The function is defined as

\begin{equation*}
 \delta(t) = \begin{cases}
              1 & t = 0 \\
              0 & t \neq 0
             \end{cases}
\end{equation*}

If the unit impulse function is given as an input ($x = \delta (t)$) to the system $\mathfrak{F}$, an output known as the unit impulse response $y = h(t)$ will be obtained.
Due to the principle of superposition, we know that if the input is shifted in time or scaled by a constant then the output will change accordingly.

\begin{align*}
 \delta (t-k) \rightarrow h(t-k) \\
 \alpha \delta (t) \rightarrow \alpha h(t)
\end{align*}

Likewise, due to additivity the following is also true

\begin{equation*}
 \alpha_1 \delta (t - k_1) +  \alpha_2 \delta (t-k_2) \rightarrow y = \alpha_1 h(t-k_1)  + \alpha_2 h(t+k_2)
\end{equation*}

An input $x(t)$ can be represented as a sum of impulse functions, each scaled accordingly for each instant of time of the function.

\begin{equation*}
 x (t) = \sum_{i = 0}^M \alpha_i \delta(t-k_i)
\end{equation*}

Likewise, the output of the system to an input $x(t)$ can be represented as the sum of the outputs for each impulse function.

\begin{equation*}
 y(t) = \sum_{i = 0}^M \alpha_i h(t-k_i)
\end{equation*}

As such, for a LTI system it is possible to fully characterize its behaviour by simply knowing the impulse response of the system.


% \printbibliography

\end{document}
