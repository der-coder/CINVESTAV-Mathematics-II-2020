\documentclass[a4paper,12pt]{article}
%\documentclass[a4paper,12pt]{scrartcl}

\usepackage{xltxtra}

\input{../preamble.tex}

\usepackage{tikz}
\usetikzlibrary{shapes,backgrounds}

% \setromanfont[Mapping=tex-text]{Linux Libertine O}
% \setsansfont[Mapping=tex-text]{DejaVu Sans}
% \setmonofont[Mapping=tex-text]{DejaVu Sans Mono}

\title{Homework \#02: Compound probability}
\author{Isaac Ayala Lozano}
\date{2020-01-15}

\begin{document}
\maketitle

\def\firstcircle{(0,0) circle (1.5cm)}
\def\secondcircle{(0:2cm) circle (1.5cm)}
\def\thirdcircle{(45:2cm) circle (1.5cm)}

Let \emph A and \emph B be two events for a sample space \emph{S}. Proofs are presented for the compund probability of $A \cup B$ employing Venn diagrams for the general case and the particular case where A and B are mutually exclusive.

\section{General case}

The probability of $A \cup B$ is given by  $P(A \cup B) = P(A) + P(B) - P(A \cap B)$. 
The probability of $A \cap B$ is subtracted because that subset is contained in both A and B ($A \cap B \subset A,  A \cap B \subset B$), hence added twice.
Thus the compound probability is obtained by adding the probability of A and B subtracted by the probability of $A \cap B$, as shown in figure 1.


\section{Particular case}

For the particular case where $A \cap B = \emptyset$, the probability of $A \cup B$ is equivalent to the sum of probabilities of A and B, subtracted by the probability of $A \cap B$ which is zero in this case:
$P(A \cup B) = P(A) + P(B) - P(A \cap B) = P(A) + P(B) - 0 = P (A) + P (B)$.

% \begin{figure}
% \centering
% \begin{tikzpicture}
%      \begin{scope}[shift={(3cm,-5cm)}, fill opacity=0.5]
%         \fill[red] \firstcircle;
%         \fill[green] \secondcircle;
%         \draw \firstcircle node[center] {$A$};
%         \draw \secondcircle node [center] {$B$};        
%     \end{scope}
% \end{tikzpicture} 
% \caption{Events A and B in the sample space S.}
% \label{fig: events}
% \end{figure}

 \begin{figure}[ht]
    \centering
    \import{./img/}{hw02-venn-diagrams.pdf_tex}
    \caption{Compound probability.}
    \label{fig: compound probability}
\end{figure}


\end{document}
