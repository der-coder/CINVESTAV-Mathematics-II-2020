\documentclass[a4paper,12pt]{article}
%\documentclass[a4paper,12pt]{scrartcl}

\usepackage{xltxtra}

\input{../preamble.tex}

% \usepackage[spanish]{babel}

% \setromanfont[Mapping=tex-text]{Linux Libertine O}
% \setsansfont[Mapping=tex-text]{DejaVu Sans}
% \setmonofont[Mapping=tex-text]{DejaVu Sans Mono}

\title{Homework \#12}
\author{Isaac Ayala Lozano}
\date{\today}

\begin{document}
\maketitle

\begin{enumerate}
 \item Propose a random process (Figure \ref{fig: samples}) and prove that it is ergodic.

 Let $y(\zeta, t) = \sum_{n=1}^N a_n \cos (\omega_n t + \Psi(\zeta))$ be the proposed random process, where

 \begin{itemize}
  \item $a_n$ is the amplitude of each cosine function.
  \item $\omega_n$ is the frequency corresponding to each $n$.
  \item $\Psi(\zeta) \in R $ is the random noise added to the process.
 \end{itemize}

 \begin{figure}[htb!]
\centering
\import{./img/}{hw12_samples.tex}
\caption{Random functions from the ensemble.}
\label{fig: samples}
\end{figure}

 We prove that the process is ergodic by verifying that for the ensemble of sample functions $y_k$, the mean value $\mu_y (k)$ and the autocorrelation function $R_{yy}(\tau, k)$ do not differ over different sample functions \cite{bendat2011random}.



 The mean value of the sample function is obtained as follows

 \begin{align*}
  \mu_y(k) &= \lim _{T \rightarrow \infty} \frac{1}{T} \int_0^T y_k (t) dt \\
  &= \lim _{T \rightarrow \infty} \frac{1}{T} \int_0^T \sum_{n=1}^N a_n \cos(\omega_n t + \Psi (\zeta)) dt\\
  &= \lim _{T \rightarrow \infty} \frac{1}{T}  \sum_{n=1}^N \int_0^T a_n \cos(\omega_n t + \Psi (\zeta)) dt\\
  &= \lim _{T \rightarrow \infty} \frac{1}{T}  \sum_{n=1}^N \left. \frac{a_n}{\omega_n} \sin(\omega_n t + \Psi (\zeta))  \right\rvert_{0}^{T} \\
  &= 0
 \end{align*}

 Given that $N < \infty$, the sum of all terms will also be number less than infinity.
 Thus, as $T$ tends towards infinity the mean value of the random process approaches zero.
 This holds true for all values of $\Psi(\zeta)$.



\begin{figure}[htb!]
\centering
\import{./img/}{hw12_ergodic.tex}
\caption{Mean vlaue for different amount of samples.}
\label{fig: ergodic}
\end{figure}

 For the autocorrelation function, a similar process is followed.

 \begin{align*}
  R_{yy} (\tau, k) & = \lim _{T \rightarrow \infty} \frac{1}{T} \int_0^T y_k (t) y_k (t + \tau) dt\\
  &= \lim _{T \rightarrow \infty} \frac{1}{T} \int_0^T \sum_{n=1}^N a_n \cos(\omega_n t + \Psi (\zeta))
  \sum_{n=1}^N a_n \cos(\omega_n (t+ \tau) + \Psi (\zeta))
  dt
 \end{align*}

We present a simplified version of the proof when $N$ is equal to one, though the proof holds for all values of $N$.
We begin by applying the trigonometric identity of $\cos(u\pm v) = \cos(u)\cos(v) \mp \sin(u) \sin(v)$ , such that $u = \omega_n t + \Psi(\zeta)$ and $v = \omega_n \tau$.

 \begin{align*}
  R_{yy} (\tau, k) &= \lim _{T \rightarrow \infty} \frac{1}{T} \int_0^T  (a \cos(u))(a\cos(u)\cos(v) - a\sin(u)\sin(v)) dt\\
  &= \lim _{T \rightarrow \infty} \frac{1}{T} \int_0^T  (a^2 (\cos (u))^2 \cos(v) - a^2 \cos(u)\sin(u)\sin(v)) dt\\
  &= \lim _{T \rightarrow \infty} \frac{a^2}{T} ( \cos(v)\int_0^T (\cos(u))^2 dt -  \sin(v) \int_0^T \cos(u) \sin(u) dt )
 \end{align*}

 Evaluating each integral yields the following results.

 \begin{align*}
\int_0^T (\cos(u))^2 dt &= \int_0^T (\cos(\omega t + \Psi(\zeta)))^2 dt \\
&= \left. \frac{2 (\omega t + \Psi(\zeta)) + \sin (2(\omega t + \Psi(\zeta)))}{4 \omega}  \right\rvert_{0}^{T}\\
&= \frac{T}{2} + \frac{\sin(2(\omega T + \Psi(\zeta)))}{4\omega} - \frac{\sin(2\Psi(\zeta))}{4\omega} \\
%
\int_0^T \cos(u) \sin(u) dt &= \int_0^T \cos(\omega t + \Psi(\zeta)) \sin(\omega t + \Psi(\zeta)) dt\\
&= \left .  - \frac{\cos (2 (\omega t + \Psi(\zeta)))}{4 \omega} \right\rvert_{0}^{T}\\
&= \frac{\cos(2\Psi (\zeta))}{4\omega} - \frac{\cos (2 (\omega T + \Psi(\zeta)))}{4 \omega}
 \end{align*}

Substituting these results into the original equation and evaluating the limit yields

\begin{equation*}
 R_{yy} (\tau, k) =  \frac{a^2}{2} \cos(\omega \tau)
\end{equation*}

This is due to the fact that every other term for the integrals lacks a $T$ in the numerator of their fractions.
Given the absence of it, as $T$ approaches infinity the value of all the other terms will approach zero.

This result can be generalized for any value of N.
From the original equation we notice that the equation was a sum of integrals, hence for values of different from one it is only necessary to add the results of those other integrals.

\begin{equation*}
 R_{yy} (\tau, k) =  \sum_{n=1}^N \frac{a_n^2}{2} \cos(\omega_n \tau)
\end{equation*}

Observe that the autocorrelation function is not dependent on time, but on the lag between measurements.
It does not vary accross sample functions because they all present the same behaviour described in the random process' equation.

Given that both $\mu_y (k)$ and $R_{yy}(k)$ do not vary between sample functions, as has been proven, the proposed random process is indeed ergodic.



\item Present plots of the probability density functions for a sine wave, a sine wave plus random noise, and a sample of white\footnote{Also called Gaussian} noise.

\item Present the autocorrelation plots for the three previous functions.

\begin{figure}[htb!]
\centering
\import{./img/}{hw12_sin.tex}
\caption{Sine function.}
\label{fig: sin}
\end{figure}

\newpage
\pagebreak

\begin{figure}[htb!]
\centering
\import{./img/}{hw12_noise.tex}
\caption{Sine function with noise.}
\label{fig: noise}
\end{figure}

\newpage
\pagebreak

\begin{figure}[htb!]
\centering
\import{./img/}{hw12_white.tex}
\caption{White noise.}
\label{fig: white}
\end{figure}

\end{enumerate}



\printbibliography

\newpage
\pagebreak
\appendix
\section{Octave Code}
\lstinputlisting[language=Matlab]{hw08_plots.m}

% https://ocw.mit.edu/courses/mechanical-engineering/2-22-design-principles-for-ocean-vehicles-13-42-spring-2005/readings/r6_spectrarandom.pdf
\end{document}
